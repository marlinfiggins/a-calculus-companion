%!TEX root = ../a-calculus-companion.tex

%% Title page with Introduction.
\maketitle


\section*{\hfill Introduction \hfill}

These notes are intended to be used as an introduction to calculus. Ideally this is meant to accompany a calculus course like that of MATH 13100-13200-13300 or MATH 15100-15200-15300 at the University of Chicago but can also be used for independent study. During my time as an undergrad at the University of Chicago, I was a TA for 6 quarters of the MATH 130s sequence. What I learned from my students was that many of them thought they were bad at mathematics. Increasingly, I began to see a divorce between theory, example, and application in the way the course was taught. My goal in these notes is to present calculus in a way that emphasizes both its reason for existing and its usefulness. I hope that by the end of this, the reader will be able to understand the language of calculus and its appreciate its usefulness. Most importantly, I wish to enable my reader to apply the ideas of this text themselves to problems they may encounter in the future.

Therefore, I will try my best to balance the theory behind calculus with my use of examples. I admit that my main purpose is to motivate calculus as a tool and develop the theory in accessible language. Most importantly, I hope provide the intuition necessary to understand calculus and its various concepts, so there may not be a ton of examples outside of those necessary to emphasize the thought process behind solving problems with calculus. In the future, I will provide exercises at the end of each section!


\thispagestyle{empty}

%% Table of Contents Page
\newpage
\tableofcontents
\thispagestyle{empty}
\newpage

%% Set first page after ToC
\setcounter{page}{1}
